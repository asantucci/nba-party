\documentclass{beamer} 
\usepackage{tikz} 
\usepackage{pdfrender} 
\usepackage{mathtools}
\usepackage{bbm}
\usepackage{hyperref}
\usepackage{changepage}

 % \usetheme{Dresden} % \usecolortheme{beaver} % \setbeamercolor{itemize item}{fg=darkred!80!black}

\setbeamertemplate{navigation symbols}{}

\usetikzlibrary{shapes,arrows} \setbeamerfont{author}{size=\Huge} 
\setbeamerfont{institute}{size=\normalsize\itshape} \setbeamerfont{title}{size=\fontsize{30}{36}\bfseries} \setbeamerfont{subtitle}{size=\Large\normalfont\slshape}  \setbeamertemplate{title page} {
  \begin{tikzpicture}[remember picture,overlay]     \fill[orange]   ([yshift=15pt]current page.west) rectangle (current page.south     east); \node[anchor=east]    at ([yshift=-50pt]current page.north east) (author)       {\parbox[t]{.6\paperwidth} {\raggedleft              \usebeamerfont{author}\textcolor{orange} {     \textpdfrender{                 TextRenderingMode=FillStroke,     FillColor=orange,     LineWidth=.1ex,               }{\insertauthor}}}};
    \node[anchor=north east]
    at ([yshift=-70pt]current page.north east) (institute) {
      \parbox[t]{.78\paperwidth} { 
        \raggedleft \usebeamerfont{institute}\textcolor{gray}         {\insertinstitute}}}; 
    \node[anchor=south west] at ([yshift=20pt]current page.west) (logo) {
      \parbox[t]{.19\paperwidth} { 
        \raggedleft \usebeamercolor[fg]{titlegraphic} \inserttitlegraphic}}; 
    \node[anchor=east] at ([yshift=-10pt,xshift=-20pt]current page.east) (title) {
      \parbox[t]{\textwidth} {
        \raggedleft \usebeamerfont{author}\textcolor{white} {  
          \textpdfrender { 
            TextRenderingMode=FillStroke, FillColor=white,             LineWidth=.1ex}{\inserttitle}}}}; \node[anchor=east] at     ([yshift=-60pt,xshift=-20pt]current page.east) (subtitle) {
      \parbox[t]{.6\paperwidth}{\raggedleft\usebeamerfont{subtitle}\textcolor{black}{\insertsubtitle}}};  
  \end{tikzpicture}
}  

\author{Andreas Santucci} \institute{Stanford Teaching Fellow \\ Computational Math, Data Science} \title{Effects of nightlife on sports} \subtitle{``Hangover Effects''} % \titlegraphic{\includegraphics[width=2cm]{ctanlion}}
 \begin{document}  \begin{frame} \maketitle \end{frame}

\begin{frame}   \frametitle{What is the problem?}
  \begin{block}{Do pro athletes party to the detriment of their performance?}     It's not unreasonable to think that what happens off the court affects game-outcomes, but!
    \begin{itemize}       
      \item Can we even measure these phenomena?
      \item Are there systematic tendencies?
    \end{itemize}       \end{block}
\end{frame}

% What is our solution?

\begin{frame}   
  \frametitle{Why is this problem hard?}
  \begin{block}{Partying is a latent variable.}
    Instead, we proxy for the possibility of athletes partying by looking at 
    play following games in party cities. Our question becomes...
  \end{block}
  \vspace{12pt}\begin{block}{Do teams exhibit a decline in performance the day following a game in a city with active nightlife?}     I.e. we are concerned with ``hangover effects''.
  \end{block}
  \vspace{12pt}\begin{block}{Part of the question is determining a measure for nightlife. }
    We colloquially call cities with high levels of nightlife ``party cities''.   \end{block}
\end{frame}

  %   \vspace{12pt} 
  %   \begin{itemize}       \item Do teams play worse the day after visiting a ``party city''?
  %     \item How do we even determine whether a city has nightlife?    
  %     \item What does it mean to ``play worse''? \end{itemize}

  % \vspace{12pt}We define ``play worse'' relative to bookmaker/public expectations.



\begin{frame}   
  \frametitle{Who cares?}
  \begin{block}{\$400 billion spent on sports betting per year in the US.\footnote{NBA Commissioner Adam Silver, 2014.}}

    Apparently, a lot of people. Aside from bettors...
  \end{block}

  \vspace{12pt}\begin{block}{The causal relationship might inform more optimal strategy.}     For example,
    % Even aside from bettors, knowing the causal relation between % partying and next-day performance might inform
    \begin{itemize}       
      \item how coaches set team-rules, 
      \item how managers determine travel schedules, as well as
      \item which players to bench, and when.\footnote{Are players more likely to get injured the next day as a result of partying?}
    \end{itemize}
  \end{block}
\end{frame}

% \begin{frame}   
%   \frametitle{Is the problem hard?}

%   \begin{block}{Finding causal-mechanisms which significantly determine % game-outcomes is difficult.}       It's not obvious whether we will find an % effect. 

%       \vspace{12pt}After all, we can't reliably observe if players party, let % alone when.

%       Even if we could, can we exploit this in the marketplace?   \end{block}

% \end{frame}

\begin{frame}
  \frametitle{How can we measure unexpected changes in performance?}     \begin{block}{Bookmaker point-spreads (NBA)}     
    The expected score differential between two teams.

    Conditions on (almost) every conceivable element of the game:
    \begin{itemize}     \item strengths and weaknesses against particular opponents,
    \item momentum,
    \item injuries, and
    \item even possibly a weather forecast.     \end{itemize}

    A team is expected to meet the spread $\frac{1}{2}$ the time.\footnote{Dubner and Levitt, ('06). Losers pay 10\% fee on top of wagers; a balanced book guarantees 5\% gain for the house-bookmaker.}
  \end{block}
  \vspace{12pt}\begin{block}{Bookmaker money-lines (MLB)}     The odds that a team will beat their opponent.

    After conditioning on odds, successful prediction is like a coin-flip. \end{block} \end{frame}

\begin{frame}
  \centering   \includegraphics[scale=0.55]{../writing/meet_the_spread} 
\end{frame}

\begin{frame}   \frametitle{Our solution: a quasi-experiment.}
  \begin{block}{Systematic deviations represent unexplained mechanisms.}     We construct a model which attempts to estimate the causal effect of visiting a ``party city''
    on subsequent day performance, as measured by the odds of meeting the spread.       \end{block}

  \begin{block}{}     We only consider the hangover effect on games played back-to-back within 24 hours of each other.   \end{block} \end{frame}

\begin{frame}   \frametitle{Identifying variation}
  \begin{block}{Performance as a function of previous day's nightlife index.}     If next day opponent is uncorrelated with visitation to a party city,
    then looking at the \emph{actual outcome} of the game $\leadsto$ identification.

    \vspace{15pt} 

    Weaker, still: if odds of beating the spread uncorrelated with previous game location
    $\leadsto$ identification.\footnote{Under an efficient market, this assumption should hold.}       \end{block}
\end{frame}


\begin{frame}
  \frametitle{Previous research - NBA player shooting average}
  \begin{block}{Empirical evidence: players party to the detriment of the game.}
  NBA Players under-perform proportion of shots made on Sundays.\footnote{Chase, ('15).}
  \end{block}
    \begin{itemize}       
    \item 1,092 of 2,035 NBA players who took $\geq 1$ shot on Sundays under-perform 
      relative to their own season-long average; $p=0.001$.
      \item Effect size is not considered.
      \item Density of support is questionable.\footnote{What if we miss the only shot we took on Sunday, is that relevant?}     \end{itemize}
\end{frame}

\begin{frame}   \frametitle{Prev. research - likelihood of meeting the spread in NBA}
  \begin{block}{Empirical evidence: away teams under-perform on Sundays.}
  Teams under-perform bookmakers expectations on Sunday afternoons when on the road.\footnote{Ezekowitz, ('14).}
    \begin{itemize}       
    \item Only 129 out of 284 games did the away team meet the spread, 
      using 2006-14 data; $p=0.1089$.
    \item Not quite significant at the 90\% confidence level.     
    \item If we bet \$100 on each Sunday game, we'd only profit \$1,310.\footnote{$(284 - 129) \times 100 + 129 \times (-110) = 1,310$.} 
    \end{itemize}
  \end{block}
  \end{frame}


% \begin{frame}   \frametitle{Our solution: a quasi-experiment}
%   Consider games played back-to-back within 24 hours.  
  
%   \vspace{12pt}
%   Examine whether spread is met as a function of nightlife index for lagged game % location.

%   \vspace{12pt} (Alcohol and loss of sleep are well correlated with nightlife; % both have been shown to have negative effects on sports % performance.)\footnote{Shirreffs and Maughan ('06), Barnes ('14), Pilcher and % Huffcutt ('96).} \end{frame}

\begin{frame}   \frametitle{Sports: NBA and MLB}
  \begin{block}{We perform the analysis for both NBA and MLB}     
    If we observe a team traveling to a city with active nightlife:
    \vspace{12pt}
    \begin{itemize}       \item For NBA teams playing another game in the next 24 hours, are they less likely to meet the points-spread?
      \item For MLB teams, after conditioning on the odds of winning, are they less likely to beat their opponent?     \end{itemize}
    \vspace{12pt}\begin{block}{If we yield a causal model, we may realize a betting schema.}     
\end{block}
  \end{block} 
\end{frame}

\begin{frame}   \frametitle{NBA data: 2010-11 through 2016-17 season}
  \begin{itemize}     \item Each team plays 82 games per season.
    \item 30 teams in the NBA.
    \item Total of $82 \times 30 / 2 = 1,230$ games per season.  
  \end{itemize}

  \vspace{12pt}We query \href{http://www.basketball-reference.com/leagues/}{basketball-reference.com}
  for a listing of game-days, and then 
  \href{http://www.covers.com/sports/NBA/matchups?selectedDate=2011-1-01}{covers.com} for 
  game-outcomes alongside a point-spread.

  \vspace{12pt}  
  \begin{itemize}     \item Missing the first couple months of 2010-11 season.
    \item At most two missing games in each of remaining seasons.
    \item Total of 7,829 games in our dataset.   \end{itemize}
\end{frame}

\begin{frame}   \frametitle{MLB data: 2011 through 2017 season}
  \begin{itemize}     \item Each team plays 162 games per season.
    \item 30 teams in the MLB.
    \item Total of $162 \times 30 / 2 = 2,430$ games per season.   \end{itemize}
  
  \vspace{12pt}We query \href{http://www.baseball-reference.com/leagues/}{baseball-reference.com}
  for a listing of game days, and again look to covers.com for game-outcomes and money-lines.
  
  \vspace{12pt}
  \begin{itemize}     \item We have on average 2,100 games per season.
    \item (Missing observations due to broken links, postponed games?)
    \item Total of 12,709 games in our dataset.    \end{itemize} \end{frame}

\begin{frame}   \frametitle{Bureau of Labor Statistics data: 2010-2016}
  We look toward BLS data which describes the number of business establishments 
  (of a very particular type) in each metropolitan statistical area
  for each quarter of each year.
  
  \vspace{12pt}  All teams are in the US with the exception of 
  \vspace{12pt}\begin{itemize}     \item Toronto Raptors (NBA)
    \item Toronto Blue Jays (MLB)   \end{itemize} 

  \vspace{12pt}These two teams are excluded from our analysis.\footnote{We don't expect Toronto to be a huge party city.} \end{frame}

\begin{frame}   \frametitle{Minutes data for NBA}
  For NBA, we also collect minutes data, which describes the number of 
  \vspace{12pt}
  \begin{itemize}     
    \item Free-throws
    \item Field goals
    \item Three-pointers   
  \end{itemize}
  \vspace{12pt}attempted and made by each player in each game. For each season and each team,
  we have nearly complete minutes data; 85\% of our season-teams have no more than 3 missing games. \end{frame}


\begin{frame}
  \frametitle{Possible confounders}
  What might influence game-outcomes apart from player-ability?

  \vspace{12pt}\begin{itemize}     
    \item Home team effect
    \item Fatigue from previous game
    \item Jet lag   
    \end{itemize}

  \vspace{12pt}The benefit of using spread data is that it controls for salient factors which affect game-outcomes.
  The downside is that we don't know exactly what it controls for. \end{frame}

\begin{frame}   \frametitle{Feature creation}
  We add features which describe
  \vspace{12pt}  
  \begin{itemize}     
    \item Number of days since the last game (i.e. rest-time),
    \item Lagged changes in possession,\footnote{A crude measure of fatigue incurred from last game.}
    \item Lagged game location, and
    \item Travel Distance.   
  \end{itemize}
  \vspace{12pt}We seek to construct a model which accounts for travel-fatigue and jet-lag.
  For this, we need a little bit more. \end{frame}

\begin{frame}   \frametitle{Accounting for jetlag}
  Prolonged air travel negatively effects sporting performance.\footnote{Lee and Galvez ('12).}

  \vspace{12pt}In order to account for jet-lag, we need to be mindful of the direction of travel. We calculate
  \vspace{12pt}\begin{itemize}     \item distance traveled between games,
    \item final direction (bearing) traveling from $a \leadsto b$, and
    \item direction of travel (in east-west degrees) by taking sine.
  \end{itemize}
  \vspace{12pt}Finally, to account for jetlag we interact this east-west measure with travel distance.\footnote{Lallensack ('17)} \end{frame}

\begin{frame}   \frametitle{Next day opponent: NBA}
  \centering \includegraphics[scale=0.25]{../writing/next_day_opponent} \end{frame}

\begin{frame}   \frametitle{A closer look at NBA games on the road}
  \centering \includegraphics[scale=0.25]{../writing/next_day_opponent_on_tour} \end{frame}

\begin{frame}   %\frametitle{Further motivation for conditioning on the spread}
  \begin{block}{With some correlation between next-day opponent, this further motivates conditioning
  on bookmakers' expectations.}
  \begin{itemize}     \item If bookkeepers aware of hangover effects, our variable of interest may not appear significant, in spite of there being a true effect.

    \item If bookkeepers systematically mis-value teams which are more likely to be encountered after visiting a city with active nightlife, we could see a spurious effect.

      (Given size of betting market, this is unlikely)
    \end{itemize}   \end{block}
\end{frame}

\begin{frame}   \frametitle{Partying as a latent variable}
  \begin{block}{We can't observe when players go for a night on the town.}
    Instead, we develop two approaches to proxy for partying. 

    A discrete indicator and a continuous measure.
  \end{block} \end{frame}

\begin{frame}   \frametitle{Discrete Indicator}
  \begin{block}{Anecdotally, LA and NY appear in news articles frequently.}     \vspace{12pt}This motivates a binary indicator for party when either of these two cities have been visited within 24-hours of another game.
\vspace{12pt}
    \[
      \texttt{party} \coloneqq \mathbbm 1 \left \{\substack{\textrm{last game played in LA or NY} \\ \textrm{at most 24-hours ago}}\right \}
    \]   \end{block} \end{frame}


\begin{frame}   
  \frametitle{Travel distance comparison}
  \centering \includegraphics[scale=0.25]{../writing/travel_dist_density_by_party} 
\end{frame}


\begin{frame}   \frametitle{Continuous measure}
  \begin{block}{A-list celebrities tend to party with other celebrities.}     Idea: proxy nightlife using number of musicians in previous location

    \[
      \texttt{party} \coloneqq \begin{cases}       \log \left(\# \textrm{musical groups}\right) &\substack{\textrm{ if last game played} \\ \textrm{at most 24-hours ago,}} \\
      0 &\textrm{ otherwise.}
    \end{cases}
    \]
    \newline
    Specifically, we consider the sum total of
    \[
      \textrm{Sound recording studios } + \textrm{ musical groups } + \textrm{ musical publishers}.
    \]   \end{block} \end{frame}

\begin{frame}   
  \frametitle{Continuous measure of party (NBA)}
  \centering \includegraphics[scale=0.3]{../writing/Party_by_Team_Location} 
\end{frame}

\begin{frame}   \frametitle{Results: NBA}
  \vspace{-2pt}
  \centering
  \tiny{ % m0, m1
\begin{tabular}{@{\extracolsep{5pt}}lcc}  \\[-1.8ex]\hline  \hline \\[-1.8ex]   & \multicolumn{2}{c}{\textit{Dependent variable:}} \\  \cline{2-3}  \\[-1.8ex] & \multicolumn{2}{c}{Meet the Spread} \\  \\[-1.8ex] & (1) & (2)\\  \hline \\[-1.8ex]   Party discrete & $-$0.557$^{***}$ &  \\    & (0.149) &  \\    & & \\   Party continuous &  & $-$0.158$^{*}$ \\    &  & (0.095) \\    & & \\   Lag changes in posession & 0.189 & 0.181 \\    & (0.140) & (0.140) \\    & & \\   Logged travel distance & $-$0.015 & $-$0.010 \\    & (0.026) & (0.026) \\    & & \\   East-west travel direction & $-$0.072 & $-$0.037 \\    & (0.234) & (0.234) \\    & & \\   Number hours rest time & 0.001 & 0.001 \\    & (0.001) & (0.001) \\    & & \\   Time of game during day & 0.006 & 0.004 \\    & (0.013) & (0.013) \\    & & \\   Home team effect & 0.002 & 0.004 \\    & (0.044) & (0.044) \\    & & \\   Logged travel distance * east-west & 0.011 & 0.007 \\    & (0.036) & (0.036) \\    & & \\   Constant & $-$0.923 & $-$0.875 \\    & (0.695) & (0.695) \\    & & \\  \hline \\[-1.8ex]  Observations & 9,517 & 9,517 \\  Log Likelihood & $-$6,586.320 & $-$6,592.159 \\  Akaike Inf. Crit. & 13,200.640 & 13,212.320 \\  \hline  \hline \\[-1.8ex]  \textit{Note:}  & \multicolumn{2}{r}{$^{*}$p$<$0.1; $^{**}$p$<$0.05; $^{***}$p$<$0.01} \\  \end{tabular} 
}   
\end{frame}

\begin{frame}   \frametitle{Placebo test (discrete indicator): what if players rest?}
  \centering
  \tiny{ % mp
\begin{tabular}{@{\extracolsep{5pt}}lc}  \\[-1.8ex]\hline  \hline \\[-1.8ex]   & \multicolumn{1}{c}{\textit{Dependent variable:}} \\  \cline{2-2}  \\[-1.8ex] & Meet the Spread (NBA) \\  \hline \\[-1.8ex]   Party placebo & 0.053 \\    & (0.084) \\    & \\   Lag changes in posession & 0.170 \\    & (0.140) \\    & \\   Logged travel distance & $-$0.004 \\    & (0.026) \\    & \\   East-west travel direction & $-$0.034 \\    & (0.235) \\    & \\   Number hours rest time & 0.001$^{*}$ \\    & (0.001) \\    & \\   Time of game during day & 0.002 \\    & (0.013) \\    & \\   Home team effect & $-$0.006 \\    & (0.044) \\    & \\   Logged travel distance * east-west & 0.006 \\    & (0.036) \\    & \\   Constant & $-$0.865 \\    & (0.695) \\    & \\  \hline \\[-1.8ex]  Observations & 9,517 \\  Log Likelihood & $-$6,593.344 \\  Akaike Inf. Crit. & 13,214.690 \\  \hline  \hline \\[-1.8ex]  \textit{Note:}  & \multicolumn{1}{r}{$^{*}$p$<$0.1; $^{**}$p$<$0.05; $^{***}$p$<$0.01} \\  \end{tabular} 
      }
\end{frame}

\begin{frame}   \frametitle{Differences between NBA and MLB}
  The same analysis can be replicated for MLB. Notable differences:
  \vspace{12pt}  
  \begin{itemize}     
    \item Money-lines are used instead of point-spreads. 
      So instead of predicting likelihood of meeting the spread,
      we predict likelihood of winning conditional on bookmakers odds.
    \item We interact our measure of nightlife with a weekend indicator.
    \item Games are played in series, so we no longer have the problem of an unknown travel schedule, i.e. players stay overnight in the same location within a series.   \end{itemize} \end{frame}

\begin{frame}   
  \frametitle{Continuous measure of party (MLB)}
  \centering \includegraphics[scale=0.3]{../writing/Party_by_Team_Location_mlb}   
\end{frame}

\begin{frame}   \frametitle{Results: MLB}
  \centering
  \tiny{ % m3, m4
\begin{tabular}{@{\extracolsep{5pt}}lcc}  \\[-1.8ex]\hline  \hline \\[-1.8ex]   & \multicolumn{2}{c}{\textit{Dependent variable:}} \\  \cline{2-3}  \\[-1.8ex] & \multicolumn{2}{c}{Probability of Winning} \\  \\[-1.8ex] & (1) & (2)\\  \hline \\[-1.8ex]   Continuous measure of nightlife & $-$0.120$^{*}$ &  \\    & (0.071) &  \\    & & \\   Nightlife (no weekend interaction) &  & 0.137 \\    &  & (0.134) \\    & & \\   Bookmaker's odds & 2.682$^{***}$ & 2.669$^{***}$ \\    & (0.158) & (0.158) \\    & & \\   Home-team effect & 0.014 & 0.144 \\    & (0.034) & (0.102) \\    & & \\   Number of rest days & 0.018 & 0.016 \\    & (0.021) & (0.021) \\    & & \\   Logged travel distance & $-$0.005 & $-$0.003 \\    & (0.005) & (0.005) \\    & & \\   Weekend & 0.044 & 0.002 \\    & (0.038) & (0.029) \\    & & \\   Constant & $-$1.363$^{***}$ & $-$1.468$^{***}$ \\    & (0.084) & (0.122) \\    & & \\  \hline \\[-1.8ex]  Observations & 26,473 & 26,473 \\  Log Likelihood & $-$18,124.350 & $-$18,125.300 \\  Akaike Inf. Crit. & 36,272.700 & 36,264.600 \\  \hline  \hline \\[-1.8ex]  \textit{Note:}  & \multicolumn{2}{r}{$^{*}$p$<$0.1; $^{**}$p$<$0.05; $^{***}$p$<$0.01} \\  \end{tabular}  }
\end{frame}

\begin{frame}   \frametitle{Validating our model in the market}
  Our results are statistically significant, but so what? We validate our model by seeing
  how well we fare against the market.

  \begin{itemize}     \item We hold out data for validation.
    \item For each unseen relevant game, we place a bet of \$100 if the expected value of the bet
      is positive after accounting for bookmakers cut (\$10)
    \item Are we profitable?
  \end{itemize} \end{frame}

\begin{frame}   \frametitle{Betting in MLB}
  \centering
  \includegraphics[scale=0.5]{../writing/mlb_bets_by_season} \end{frame}

\begin{frame}   \frametitle{Returning to NBA: player performance}
  We additionally look toward actual performance metrics, instead of relying on point-spread data.
  
  \vspace{12pt}
  \begin{itemize}     
    \item Avoid using black-box feature
    \item Pinpoint more precisely what elements of game-play are affected by nightlife visitation.   
\end{itemize}

\vspace{12pt}  Anecdotally, rebounds are considered a ``hustle'' statistic, in the sense that players
  who work harder on the court and expend energy boxing out other players will have higher
  rebound rates. \end{frame}

\begin{frame}   \frametitle{Analysis: rebounds}
  \begin{itemize}     \item For each player and each game, we calculate the number of rebounds obtained per-minute of game-play
    \item From each observation, we subtract the player's average number of rebounds per minute taken across the entire season.
    \item We then run a linear regression explaining demeaned rebounds as a function of 
      nightlife visitation and other features describing player fatigue.   \end{itemize} \end{frame}

\begin{frame}   \frametitle{Results: rebounds}
  \centering
  \tiny{ % m2
\begin{tabular}{@{\extracolsep{5pt}}lc}  \\[-1.8ex]\hline  \hline \\[-1.8ex]   & \multicolumn{1}{c}{\textit{Dependent variable:}} \\  \cline{2-2}  \\[-1.8ex] & Demeanead Rebounds Per Minute \\  \hline \\[-1.8ex]   Continuous party measure & $-$0.004$^{*}$ \\    & (0.002) \\    & \\   Lag change in possessions & 0.00002 \\    & (0.0001) \\    & \\   Logged travel distance & 0.001 \\    & (0.001) \\    & \\   Number hours since last game & 0.00002 \\    & (0.00002) \\    & \\   Constant & $-$0.007 \\    & (0.004) \\    & \\  \hline \\[-1.8ex]  Observations & 40,684 \\  R$^{2}$ & 0.0003 \\  Adjusted R$^{2}$ & 0.0002 \\  Residual Std. Error & 0.104 (df = 40679) \\  F Statistic & 2.696$^{**}$ (df = 4; 40679) \\  \hline  \hline \\[-1.8ex]  \textit{Note:}  & \multicolumn{1}{r}{$^{*}$p$<$0.1; $^{**}$p$<$0.05; $^{***}$p$<$0.01} \\  \end{tabular}  } \end{frame}

\begin{frame}   \frametitle{Hangover correlates with allowing more points}
  We examine points allowed and points scored (on the team level).

  \vspace{12pt}Common knowledge suggests that overall defense of a team is strongly influenced by individual player effort.

  \vspace{12pt}Is it the case the hangover effects are more prevalent on defensive aspects of gameplay? \end{frame}

\begin{frame}   \frametitle{Defense in the NBA}
  \begin{adjustwidth}{-.75cm}{}
  \tiny{  % mpa, tpa, tps
\begin{tabular}{@{\extracolsep{5pt}}lccc}  \\[-1.8ex]\hline  \hline \\[-1.8ex]   & \multicolumn{3}{c}{\textit{Dependent variable:}} \\  \cline{2-4}  \\[-1.8ex] & Points Admitted by Team & Team Points Admitted & Team Points Scored \\  \\[-1.8ex] & (1) & (2) & (3)\\  \hline \\[-1.8ex]   Discrete party indicator & 2.817$^{***}$ &  &  \\    & (0.712) &  &  \\    & & & \\   Continuous party measure &  & 1.942$^{***}$ & 0.377 \\    &  & (0.248) & (0.246) \\    & & & \\   Lag change in possessions &  & 0.022$^{***}$ & 0.035$^{***}$ \\    &  & (0.007) & (0.007) \\    & & & \\   Logged travel distance &  & $-$0.411$^{***}$ & 0.035 \\    &  & (0.074) & (0.073) \\    & & & \\   Number hours since last game &  & 0.012$^{***}$ & 0.020$^{***}$ \\    &  & (0.003) & (0.003) \\    & & & \\   Constant & $-$0.046 & 102.824$^{***}$ & 99.540$^{***}$ \\    & (0.091) & (0.483) & (0.478) \\    & & & \\  \hline \\[-1.8ex]  Observations & 15,139 & 44,689 & 44,689 \\  R$^{2}$ & 0.001 & 0.002 & 0.002 \\  Adjusted R$^{2}$ & 0.001 & 0.002 & 0.002 \\  Residual Std. Error & 11.137 (df = 15137) & 12.117 (df = 44684) & 12.015 (df = 44684) \\  F Statistic & 15.670$^{***}$ (df = 1; 15137) & 27.968$^{***}$ (df = 4; 44684) & 24.325$^{***}$ (df = 4; 44684) \\  \hline  \hline \\[-1.8ex]  \textit{Note:}  & \multicolumn{3}{r}{$^{*}$p$<$0.1; $^{**}$p$<$0.05; $^{***}$p$<$0.01} \\  \end{tabular}      }
\end{adjustwidth} 
\end{frame}

\begin{frame}   \frametitle{Future research - NBA}
  Within the NBA
  \begin{itemize}     
    \item Are teams more likely to party after a win or a       loss?
    \item Do teams who aren't going to make the playoffs party more?
    \item Are there demographic differences which influence propensity to party? E.g. are younger, unmarried players more susceptible to our treatment effect?
    \item We could also examine how shots taken vs. shots scored are affected for star players.
  \end{itemize}
  \vspace{12pt}  A preliminary and ad-hoc analysis shows that injury risk increases as a function of party.\footnote{We look for instances where players are put on the bench and stop playing (at least several) games as a function of our nightlife indicator.}
  This might inform coaches which players to bench and when.

\end{frame}

\begin{frame}   \frametitle{Future research (MLB and general)}
  \begin{itemize}     
  \item Within the MLB, we can look at statistics such as batting averages or general fielding statistics. 
    \item Possibly prolonged effects; recall our magic 24 hour number.
    \item European soccer leagues.   \end{itemize} \end{frame}

\begin{frame}   \frametitle{Conclusions}
  \begin{itemize}     \item NBA and MLB players under-perform bookmakers expectations after visiting LA and NY, and in general we see that performance decreases as a function of (our proxy for) nightlife.
    \item The decrease in likelihood can (in part) be explained by a drop in defensive performance.
    \item With a causal model established, we realize a profitable betting scheme.   \end{itemize} \end{frame}

\begin{frame}   \frametitle{Why \# drinking establishments or population won't work}
  \includegraphics[scale=0.3]{../writing/why_drinks_or_population_dont_work} \end{frame}

\end{document}


