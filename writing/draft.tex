%% This file is to be used as a template for your submission. 
%% Rename this file and replace the text with the text of 
%% your manuscript.
%%
%% The standard LaTeX document class "article" is recommended. 
%% Use options letterpaper and 12pt.
\documentclass[letterpaper,12pt]{article}

%% This is the recommended preamble for your document.

%% Load De Gruyter specific settings 
\usepackage{dgjournal}          

\usepackage{hyperref}

%% The mathptmx package is recommended for Times compatible math symbols.
%% Use mtpro2 or mathtime instead of mathptmx if you have the commercially
%% available MathTime fonts.
%% Other options are txfonts (free) or belleek (free) or TM-Math (commercial)
\usepackage{mathptmx}

%% Use the graphics package to include figures
\usepackage{graphics}

%% Use natbib with these recommended options
\usepackage[authoryear,comma,longnamesfirst,sectionbib]{natbib} 

%% Start your document body here
\begin{document}

%% Do NOT include any fronmatter information; including the title, author names,
%% institutes, acknowledgments and title footnotes (author information, funding
%% sources, etc.). Start the document with the first section or paragraph of
%% the article.

\section{Introduction}
It's not unreasonable to think that game performance can be affected 
(in part) by what takes place off the court. We are interested to see if 
teams exhibit a decline in performance the day following a game in a city 
with active nightlife. We exploit data on bookmaker spreads, the expected score 
differential between two teams after conditioning on observable performance. 
We expect a team to meet the spread half the time, since this is how bookmakers 
minimize risk on their end. We construct a model which attempts to 
estimate the causal effect of  visiting a "party-city" on subsequent day 
performance as measured by the spread. Since next-day
opponent is uncorrelated with exposure to treatment, we have identification in our variable of 
interest.

Citations toward anecdotal. % The ``JR Smith'' effect.

\section{Previous Research}
Studies on effects of alcohol on sports performance in a controlled setting.

\section{Materials and Methods}
Determining when players partied. One might consider to look at Twitter data
and or night clubs per city. 

Insights: major party cities are LA and NY, and athletes party is well correlated with musicians. This led to two separate definitions of party. Our most basic analysis was to consider performance 
the day after visiting LA or NY.
We also considered the number of musical establishments per city. 

\paragraph{Logic for using spread data}
The logic for using spread data: it controls for a lot, but we don't know exactly
what it controls for.

\subsection{Data}
\subsubsection{Lines data} We obtained data from \href{http://www.covers.com/sports/NBA/matchups?selectedDate=2011-1-01}{Covers.com}, which records the outcome of each game alongside the line set by a betting house.


\subsubsection{BLS data} We also obtained data from
\href{https://www.bls.gov/data/}{Bureau of Labor Statistics} which records the number of establishments by business type at the county-quarter level. As a 
proxy for how much night-life there is in a city, we look toward the 
number of sound recording studios, musical groups, and music publishers there 
are for a particular county-year: we simply calculate the average number of establishments across all three music categories listed above.\footnote{We caveat that since the Toronto Raptors are located in Canada, we don't have BLS data for this team.}

We then merge this in with our lines data, taking care to do so according to each season-team's last game location; we also lag our BLS data by one year.
To this data, we've created features such as 
the number of days since last game, the last game location, the travel distance, and
average age\footnote{We use $\log \textrm{\# days old}$, at the time the game is played.}. Our data range from 2010-2011 season through present, which includes part of the 2016-2017 season.

\subsubsection{Minutes data}
We also gathered minutes data from \href{http://www.espn.com/nba/scoreboard/_/date/}{ESPN}.

The minutes data also contain information on the number of free throws, field goals, and
three pointers made and attempted by each player in each game.

\subsubsection{Which cities (don't) have nightlife?}

We examine which cities have a notable nightlife, as measured by our proxy;
bottom and top five.

Los Angeles stands out, as does New York. Some runner ups include Chicago and Miami. 

\subsection{Method: Defining a continuous measure of night life}

We have a free parameter: the time-window which we believe the hangover effect
to last. Suppose we only look toward games which have been played in a party
city at most 24 hours ago. We take care to ignore
teams playing in their own cities.

\subsection{Possible confounders}
What if we pick up effects of a busy play schedule?
Aside from accounting for how much rest time a team is allowed in between games,
and travel distance between games, we're also interested to get a measure of
fatigue incurred on the court.
We account for this by bringing in minutes data, and counting the number of 
times the players run back and forth on the court. This is crudely measured
by the number of changes in posession. We back out this quantity.

\section{Results}

\subsection{Extending the research}
We can also validate where points are lost in MLB in a similar way.

\subsection{Betting performance}


\section{Conclusions}

\section{References}




\section{This should be the first section of your document}

Your text goes here. Start with the first section or paragraph of your
article. Do not set use either headers or footers and do not set any running
heads or change any page numbers.

The title page along with headers and footers will be inserted by the EdiKit
system. Use the ``revise manuscript'' link to enter this information in the
EdiKit system.

Use the standard LaTeX commands to set the text of the article. The dgjournal
package works with any of the standard LaTeX document classes - article, report
or book. Other document classes should work but compliance with De Gruyter
requirements is not assured. For more information on LaTeX, see
\cite{lamport,mittelbach,oetiker}. \cite{mittelbach} is highly recommended.

Use the standard LaTeX sectioning commands for your headings.

\subsection{A second order heading}

Some text under the subheading. Paragraphs that follow heads are not
indented.

Math should also be set in Times. Use the mathptmx package if you do not have
any of the commercially available fonts that are compatible with Times.
\begin{equation}
    y^{(n)} = \sum_{i=0}^{n-1} a_i(x) y^{(i)} + r(x) 
\end{equation}

All environments provided by the standard LaTeX document classes are
unchanged. Vertical spaces within lists have been altered to comply with De Gruyter
requirements.
\begin{enumerate}
\item This is the first item within the list. Some more text here in order to
  display the alignment.
\item Another item in the list.
\item Yet another item in the list.
\end{enumerate}

Here is an example of a Figure. It's the same as in standard LaTeX.

\begin{figure}[!h]
%% Use the graphics package to insert figures
%% \includegraphics{figure.eps}
% Use \centering to center the table
\centering
%% A small box in place of a figure
\framebox{%
  \begin{minipage}{10pc}
    \begin{center}
      \vspace{1cm}\par
      A figure\par
      \vspace{1cm}
    \end{center}
  \end{minipage}}
\caption{Insert your caption here. If you wish to label your figure for
  cross-referencing, use a label either within the caption or after it.}
\label{fig1}
\end{figure}

An example of a table follows. This is also the same as in standard LaTeX.

\begin{table}[!h]
% Use \centering to center the table
\centering
\caption{Insert your table caption here. If you wish to label the table for
  cross-referencing, use a label either within the caption or after it.}
\begin{tabular}{llll}
\hline
Symbol        & LaTeX Command      & Symbol      & LaTeX Command \\
\hline
$\alpha$      & \verb+\alpha+      & $\zeta$     & \verb+\zeta+ \\
$\beta$       & \verb+\beta+       & $\eta$      & \verb+\eta+ \\
$\gamma$      & \verb+\gamma+      & $\theta$    & \verb+\theta+ \\
$\delta$      & \verb+\delta+      & $\vartheta$ & \verb+\vartheta+ \\
$\epsilon$    & \verb+\epsilon+    & $\iota$     & \verb+\iota+ \\
$\varepsilon$ & \verb+\varepsilon+ & $\kappa$    & \verb+\kappa+ \\
\hline
\end{tabular}
\end{table}

Use the \verb+thebibliography+ environment for the references.  BibTeX users may
use the provided BibTeX style file DeGruyter.bst.

%% BibTeX support
\bibliographystyle{DeGruyter}
\bibliography{sample}

\end{document}