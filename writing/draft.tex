%% This file is to be used as a template for your submission. 
%% Rename this file and replace the text with the text of 
%% your manuscript.
%%
%% The standard LaTeX document class "article" is recommended. 
%% Use options letterpaper and 12pt.
\documentclass[letterpaper,12pt]{article}

%% This is the recommended preamble for your document.

%% Load De Gruyter specific settings 
\usepackage{dgjournal}          

\usepackage{hyperref}
\usepackage{color,soul}

%% The mathptmx package is recommended for Times compatible math symbols.
%% Use mtpro2 or mathtime instead of mathptmx if you have the commercially
%% available MathTime fonts.
%% Other options are txfonts (free) or belleek (free) or TM-Math (commercial)
\usepackage{mathptmx}

%% Use the graphics package to include figures
\usepackage{graphics}

%% Use natbib with these recommended options
\usepackage[authoryear,comma,longnamesfirst,sectionbib]{natbib} 

%% Start your document body here
\begin{document}

%% Do NOT include any fronmatter information; including the title, author names,
%% institutes, acknowledgments and title footnotes (author information, funding
%% sources, etc.). Start the document with the first section or paragraph of
%% the article.

\section{Introduction}
It's not unreasonable to think that game performance can be affected 
(in part) by what takes place off the court. 
Anecdotally, it has been reported that professional sports teams party on 
the road, sometimes to the detriment of the game.\cite{hoch}
Within the NBA, J.R. Smith is infamous for citing his improvement in performance due to a
lack of nightlife when moving to Cleveland from New York.\cite{price,ley}
A few interviews with former professional athletes revealed that location
matters more than day of the week, i.e. parties can happen any day of the week,
but they are more likely to occur in select cities.

We are interested to see if 
teams exhibit a decline in performance the day following a game in a city 
with active nightlife. We exploit data on bookmaker spreads, the expected score 
differential between two teams after conditioning on observable performance. 
We expect a team to meet the spread half the time, since this is how bookmakers 
minimize risk on their end. We construct a model which attempts to 
estimate the causal effect of  visiting a ``party-city'' on subsequent day 
performance as measured by the spread. Since next-day
opponent is uncorrelated with exposure to treatment, we have identification in our variable of 
interest.

\section{Previous Research}
There has been an attempt to answer this question before.\cite{ezekowitz} 
However, the study only
looks at performance in party cities, as opposed to subsequent day performance (i.e. the study does not consider a delayed hangover effect). Moreover, their sample size was much smaller. 
Finally, there was no methodology used behind the selection of party cities. The author does conclude
that teams systematically play worse on Sunday afternoons when on the road,
regardless of cities.

Studies have also been carried out on statistical ranking of the players
most notorious for pursuing active nightlife in the NBA.\cite{chase} The study
does conclude that players perform worse on Sundays when compared with other days
as measured by the ratio of shots scored to shots taken.

Studies on effects of alcohol on sports performance in a controlled setting
show that alcohol impairs the nervous system, resulting in decrements in both
cognitive function and motor skill.\cite{shirreffs} In a study on 
the effects of alcohol on recovery of male athletes, it was found that
consumption is correlated with reduced production of testosterone
which ultimately leads to inhibited ability to recover and adapt to exercise.\cite{mjbarnes}
Sleep deprivation has also been shown to strongly impair cognitive and motor performance.\cite{pilcher} Both alcohol and sleep deprivation are well correlated with lifestyles associated
with an active night life.

\section{Materials and Methods}
One of the hardest parts of this analysis is that partying is a latent variable.
In order to determine when players partied, one might consider to look at social media data
for direct mentions of partying, either by athletes or witnesses. However, due to
intense public scrutiny that athletes face, there is a strong reluctance to post
such mentions on social media. There are notable exceptions, such
as Oddel Beckham Jr.'s trip to Miami in 2017 season.\cite{bleler}
The reports are few and far enough between,
and whether they are even reported at all is strongly biased on whether a particular city 
has nightclubs with VIP rooms, etc; picking up statistical evidence would be fruitless in this case.

\paragraph{Two approaches: a discrete indicator and a continuous measure of nightlife}
Because famous athletes often enjoy a celebrity lifestyle, using conventional statistics
such as number of drinking establishments in each city does not work; athletes instead
tend to visit high end nightclubs. We again turned to player interviews for ideas,
and found two major insights. From our conversations, 
we obtained a listing of cities which were anecdotally
mentioned as having active night-life. Between players, the rank ordering of these
was consistent to the extent that LA and NY were top
offenders. This motivates using a binary indicator for party in these two cities. 
We also found that athletes tend to party with other A-list celebrities,
particularly musicians. This motivated a continuous measure of nightlife, based on a
simple count of the number of musicians in the county in which the stadium resides. 
We pursue both approaches.


\paragraph{Logic for using spread data}
In order to assert a causal relation between cities with active nightlife 
and subsequent day sports performance, we need to control for salient
features such as the ability of players on either team in the match-up,
whether any players are injured, the momentum of success each team is carrying
in the season, etc. To that end, we utilize spread data. The benefit of
using spread data is that it controls for a lot, but the downside is that
we don't know exactly what it controls for. 

For example, if book-keepers
are already aware of the effects of night-life on performance, it could
already be factored into the spread; if this were the case, then our variable
of interest would not appear to have significant explanatory power
after conditioning on spread, in spite of there being
a true effect. Alternatively,
if the house book-keeper systematically over-valued
teams which you are more likely to play after a party city,
we could see a spurious effect.

\hl{Careful} That said, the amount of money that is at stake and the effort put into creating these spreads 
suggests that they are reliable. So, if we do find an effect in our variable of interest,
we can be confident in our conclusion that there is a causal effect and not some
other systematic bias.

\subsection{Possible confounders}

\paragraph{Home team effect}
Home-team advantage is well-studied.\cite{jones07, jones08}  
Because teams play a lot of their games on the road,
most games after playing in a party city will be an away game.
This should be covered by spread data, but we also include
a home-team indicator to be sure.

\paragraph{Fatigue from previous game}
What if we pick up effects of a busy play schedule?
Aside from accounting for how much rest time a team is allowed in between games,
we're also interested to get a measure of
fatigue incurred on the court.
We account for this by bringing in minutes data, and counting the number of 
times the players run back and forth on the court. This is crudely measured
by the number of changes in posession. We back out this quantity.

\paragraph{Fatigue from Jet lag}
What about jet-lag? Prolonged air travel has in general been shown
to have a negative effect on sports performance.\cite{leeandgalvez}
Previous studies have shown that jet lag is a strong enough effect
to erase home field advantage in MLB.\cite{songetal}

\paragraph{The problem of an unknown travel schedule} In NBA, we have the difficulty of
not observing when teams travel. 
In MLB, each match-up is actually a series of games.


\subsection{Data}
\subsubsection{Lines data} We obtained data from \href{http://www.covers.com/sports/NBA/matchups?selectedDate=2011-1-01}{Covers.com}, which records the outcome of each game alongside the line set by a betting house.


\subsubsection{BLS data} We also obtained data from
\href{https://www.bls.gov/data/}{Bureau of Labor Statistics} which records the number of establishments by business type at the county-quarter level. As a 
proxy for how much night-life there is in a city, we look toward the 
number of sound recording studios, musical groups, and music publishers there 
are for a particular county-year: we simply calculate the average number of establishments across all three music categories listed above.\footnote{We caveat that since the Toronto Raptors are located in Canada, we don't have BLS data for this team.}

We then merge this in with our lines data, taking care to do so according to each season-team's last game location; we also lag our BLS data by one year.
To this data, we've created features such as 
the number of days since last game, the last game location, the travel distance, and
average age\footnote{We use $\log \textrm{\# days old}$, at the time the game is played.}. Our data range from 2010-2011 season through present, which includes part of the 2016-2017 season.

\subsubsection{Minutes data}
We also gathered minutes data from \href{http://www.espn.com/nba/scoreboard/_/date/}{ESPN}.

The minutes data also contain information on the number of free throws, field goals, and
three pointers made and attempted by each player in each game.

\subsubsection{Loctaion and Travel Distance Data} is obtained
by API.

\subsection{Method: Defining a continuous measure of night life}

We have a free parameter: the time-window which we believe the hangover effect
to last. Suppose we only look toward games which have been played in a party
city at most 24 hours ago. We take care to ignore
teams playing in their own cities.

\paragraph{Discrete indicator}
If a team plays in Los Angeles or New York within 24 hours of their current game,
we indicate one, else zero.

\paragraph{Continuous measure}
If the last game was played at most 36 hours ago, use log average number
of musicians in last game location, else zero.

\subsection{Model} We use a logistic regression to estimate the likelihood of meeting
the spread as a function of whether the team partied as well as controlling for other confounders
listed above.

\section{Results}
\subsection{Using spread data}
\begin{center}
\begin{tabular}{@{\extracolsep{5pt}}lc}  \\[-1.8ex]\hline  \hline \\[-1.8ex]  \\[-1.8ex] & outcome == "W" \\  \hline \\[-1.8ex]   party & $-$0.052$^{**}$ (0.022) \\    lag.chg.pos & 0.0001 (0.002) \\    travel.dist & $-$0.00002 (0.00004) \\    ew & $-$0.035 (0.047) \\    nhours.lgame & 0.001 (0.001) \\    I(hour(date)) & 0.006 (0.012) \\    travel.dist:ew & 0.00004 (0.00005) \\    Constant & $-$0.144 (0.348) \\   \textit{N} & 9,979 \\  Log Likelihood & $-$6,912.426 \\  Akaike Inf. Crit. & 13,840.850 \\  \hline  \hline \\[-1.8ex]  \textit{Notes:} & \multicolumn{1}{r}{$^{***}$Significant at the 1 percent level.} \\   & \multicolumn{1}{r}{$^{**}$Significant at the 5 percent level.} \\   & \multicolumn{1}{r}{$^{*}$Significant at the 10 percent level.} \\  \end{tabular} 
\end{center}


\subsection{Using performance statistics}
We additionally looked at actual performance instead of relying on spread data. 
This allows us to avoid using the black-box of spread data, and also pin-point
more exactly what elements of the game are in fact effected by partying.
This motivates looking at number of points allowed, number of points scored.

We also considered rebounds, fouls, and even injuries. 
\begin{center}
    \begin{tabular}{@{\extracolsep{5pt}}lc}  \\[-1.8ex]\hline  \hline \\[-1.8ex]  \\[-1.8ex] & demeaned.reb.per.min \\  \hline \\[-1.8ex]   party & $-$0.001$^{*}$ (0.0005) \\    lag.chg.pos & 0.00003 (0.0001) \\    travel.dist & 0.00000$^{**}$ (0.00000) \\    nhours.lgame & 0.00002 (0.00002) \\    Constant & $-$0.003$^{**}$ (0.002) \\   \textit{N} & 44,190 \\  R$^{2}$ & 0.0003 \\  Adjusted R$^{2}$ & 0.0003 \\  Residual Std. Error & 0.105 (df = 44185) \\  F Statistic & 3.857$^{***}$ (df = 4; 44185) \\  \hline  \hline \\[-1.8ex]  \textit{Notes:} & \multicolumn{1}{r}{$^{***}$Significant at the 1 percent level.} \\   & \multicolumn{1}{r}{$^{**}$Significant at the 5 percent level.} \\   & \multicolumn{1}{r}{$^{*}$Significant at the 10 percent level.} \\  \end{tabular}  
\end{center}



\subsection{Robustness check}
We verify that this phenomena persists in the MLB. In MLB, money-lines
are used instead of point-spreads. So as opposed to predicting
the likelihood of meeting the spread, we predict the probability
a team would win against their opponent as both a function of the book-makers'
odds of winning and party. Additionally, we interacted weekend.
One of the major differences between NBA and MLB games is that MLB games
are played in series, so the same match-up repeated multiple times. This remedies
the problem of an unknown travel schedule we faced with NBA because we are sure 
teams stay overnight in the same location within a series.

\begin{center}
    \begin{tabular}{@{\extracolsep{5pt}}lc}  \\[-1.8ex]\hline  \hline \\[-1.8ex]  \\[-1.8ex] & I(team.score \textgreater  opponent.score) \\  \hline \\[-1.8ex]   party & $-$0.034$^{*}$ (0.019) \\    odds & 2.602$^{***}$ (0.162) \\    I(team == location) & 0.028 (0.033) \\    ndays.lgame & 0.012 (0.019) \\    travel.dist & 0.00000 (0.00003) \\    weekend & 0.042 (0.036) \\    Constant & $-$1.333$^{***}$ (0.080) \\   \textit{N} & 25,215 \\  Log Likelihood & $-$17,274.460 \\  Akaike Inf. Crit. & 34,562.920 \\  \hline  \hline \\[-1.8ex]  \textit{Notes:} & \multicolumn{1}{r}{$^{***}$Significant at the 1 percent level.} \\   & \multicolumn{1}{r}{$^{**}$Significant at the 5 percent level.} \\   & \multicolumn{1}{r}{$^{*}$Significant at the 10 percent level.} \\ \end{tabular}  
\end{center}


\subsubsection{Betting performance}
\begin{figure}[!h]
  \centering
  \includegraphics{mlb_bets_by_season}
  \caption{Betting performance by season. In each season, The 
    circled point represents the worst out of pocket expense incurred during the season.
    We first train a model on historical data then use said model to predict on ``current year''
    data, ploting profits as a function of time over the ``current'' season. For each relevant
    game, we place a bet if our model instructs us of odds sufficiently different from what
    the betting house suggests. We account for the bookmaker's cut. E.g. in the early 2016 season,
    at its worst we are out of pocket \$750 dollars. By season end, we have recouped all of
    our initial investment plus an additional \$11.5k.}
  \label{bettingperf}
\end{figure}


\subsection{Extending the research}
We can also validate where points are lost in MLB in a similar way.
It would be interesting to look at number of shots taken, and number of shots taken by stars.
It's possible there are prolonged effects of partying, and applying some sort of kernel
on the number of hours played since the last game could be interesting. It could also be
that different (types of) teams or players are more or less susceptible to the effects
of night-life; this could be demographic characteristics such as age or marital status.

\section{Conclusions}

\section{References}




\section{This should be the first section of your document}

Your text goes here. Start with the first section or paragraph of your
article. Do not set use either headers or footers and do not set any running
heads or change any page numbers.

The title page along with headers and footers will be inserted by the EdiKit
system. Use the ``revise manuscript'' link to enter this information in the
EdiKit system.

Use the standard LaTeX commands to set the text of the article. The dgjournal
package works with any of the standard LaTeX document classes - article, report
or book. Other document classes should work but compliance with De Gruyter
requirements is not assured. For more information on LaTeX, see
\cite{lamport,mittelbach,oetiker}. \cite{mittelbach} is highly recommended.

Use the standard LaTeX sectioning commands for your headings.

\subsection{A second order heading}

Some text under the subheading. Paragraphs that follow heads are not
indented.

Math should also be set in Times. Use the mathptmx package if you do not have
any of the commercially available fonts that are compatible with Times.
\begin{equation}
    y^{(n)} = \sum_{i=0}^{n-1} a_i(x) y^{(i)} + r(x) 
\end{equation}

All environments provided by the standard LaTeX document classes are
unchanged. Vertical spaces within lists have been altered to comply with De Gruyter
requirements.
\begin{enumerate}
\item This is the first item within the list. Some more text here in order to
  display the alignment.
\item Another item in the list.
\item Yet another item in the list.
\end{enumerate}

Here is an example of a Figure. It's the same as in standard LaTeX.

\begin{figure}[!h]
%% Use the graphics package to insert figures
%% \includegraphics{figure.eps}
% Use \centering to center the table
\centering
%% A small box in place of a figure
\framebox{%
  \begin{minipage}{10pc}
    \begin{center}
      \vspace{1cm}\par
      A figure\par
      \vspace{1cm}
    \end{center}
  \end{minipage}}
\caption{Insert your caption here. If you wish to label your figure for
  cross-referencing, use a label either within the caption or after it.}
\label{fig1}
\end{figure}

An example of a table follows. This is also the same as in standard LaTeX.

\begin{table}[!h]
% Use \centering to center the table
\centering
\caption{Insert your table caption here. If you wish to label the table for
  cross-referencing, use a label either within the caption or after it.}
\begin{tabular}{llll}
\hline
Symbol        & LaTeX Command      & Symbol      & LaTeX Command \\
\hline
$\alpha$      & \verb+\alpha+      & $\zeta$     & \verb+\zeta+ \\
$\beta$       & \verb+\beta+       & $\eta$      & \verb+\eta+ \\
$\gamma$      & \verb+\gamma+      & $\theta$    & \verb+\theta+ \\
$\delta$      & \verb+\delta+      & $\vartheta$ & \verb+\vartheta+ \\
$\epsilon$    & \verb+\epsilon+    & $\iota$     & \verb+\iota+ \\
$\varepsilon$ & \verb+\varepsilon+ & $\kappa$    & \verb+\kappa+ \\
\hline
\end{tabular}
\end{table}

Use the \verb+thebibliography+ environment for the references.  BibTeX users may
use the provided BibTeX style file DeGruyter.bst.

%% BibTeX support
\bibliographystyle{DeGruyter}
\bibliography{sample}

\end{document}